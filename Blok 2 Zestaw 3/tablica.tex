\documentclass[a4paper,12pt]{article}
%\documentclass[a4paper,12pt]{scrartcl}

\usepackage[utf8x]{inputenc}
\usepackage{polski}

\title{Zadania domowe. Blok 2. Zestaw 3}
\author{Maciej Poleski}
\usepackage{amsmath}
\usepackage{amsfonts}
\usepackage{amssymb}
\usepackage{alltt}
\usepackage{listings}

\date{\today}

\pdfinfo{%
  /Title    (Zadania domowe. Blok 2. Zestaw 3)
  /Author   (Maciej Poleski)
  /Creator  (Maciej Poleski)
  /Producer (Maciej Poleski)
  /Subject  (MP)
  /Keywords (MP)
}

\begin{document}
\maketitle

\newpage

\section{Ścieżki}
Przedstawię funkcję \verb|relax| którą można zastosować w standardowych algorytmach typu Dijkstra.
\begin{alltt}
 struct Edge
 \{
     bool isRed;
     int w;  // waga
     int v;  // wierzchołek docelowy (reprezentacja kursorowa)
      ...
 \};
 struct Distance
 \{
     int withRed;
     int withoutRed;
 \};
 Distance *distance;
 
 bool relax(int v, Edge e)
 \{
     if(e.isRed)
     \{
         if(distance[e.v].withRed>distance[v].withoutRed+e.w)
         \{
             distance[e.v].withRed=distance[v].withoutRed+e.w;
             return true;
         \}
         else
             return false;
     \}
     else
     \{
         bool result=false;
         if(distance[e.v].withoutRed>distance[v].withoutRed+e.w)
         \{
             distance[e.v].withoutRed=distance[v].withoutRed+e.w;
             result=true;
         \}
         if(distance[e.v].withRed>distance[v].withRed+e.w)
         \{
             distance[e.v].withRed=distance[v].withRed+e.w;
             result=true;
         \}
         return result;
     \}
 \}
\end{alltt}
Każdy element tablicy \verb|distance| odpowiada wierzchołkowi o danym numerze. Zakładam, że początkowo wszystkie odległości to $\infty$. Rozwiązanie problemu to minimum z wartości pól w strukturze \verb|Distance| odpowiadającej wybranemu wierzchołkowi docelowemu po zakończeniu algorytmu.

\end{document}
