\documentclass[a4paper,12pt]{article}
%\documentclass[a4paper,12pt]{scrartcl}

\usepackage[utf8x]{inputenc}
\usepackage{polski}

\title{Zadania domowe. Blok 4. Zestaw 1}
\author{Maciej Poleski}
\usepackage{amsmath}
\usepackage{amsfonts}
\usepackage{amssymb}
\usepackage{alltt}
\usepackage{listings}

\date{\today}

\pdfinfo{%
  /Title    (Zadania domowe. Blok 4. Zestaw 1)
  /Author   (Maciej Poleski)
  /Creator  (Maciej Poleski)
  /Producer (Maciej Poleski)
  /Subject  (MP)
  /Keywords (MP)
}

\begin{document}
\maketitle

\newpage

\section{Najbliżsi}
Zbiór kluczy znajduje się w tablicy \verb|input[n]|. Dodatkowo dostępna jest zmienna \verb|k|. W przybliżeniu jeżeli tablica \verb|input| była by posortowana to poszukiwalibyśmy elementów o indeksach z przedziału $[\frac{n}{2}-\frac{k}{2};\frac{n}{2}+\frac{k}{2}]$. W przybliżeniu, ponieważ pozycja przedziału zależy od parzystości $n$ i $k$. Możemy więc znaleźć kresy tego przedziału przy użyciu \verb|kthElement| i przefiltrować tablicę wybierając z niej te elementy, których wartość znajduje się w odpowiednim przedziale. Dla niektórych danych wejściowych odpowiedź jest niejednoznaczna, poniższe rozwiązanie wskaże przykładową odpowiedź.
\begin{alltt}
template<typename T>
T* solution()
\{
    T* result=new T[k];
    T* i=result;
    T left=kthElement(input,n/2-k/2,n);
    T right=kthElement(input,n/2-k/2+k-1,n);
    for(T o : input)
        if(left<=o && o<=right)
            *i++=o;
    return result;
\}
\end{alltt}
\verb|T| jest typem elementów w tablicy wejściowej. Wynikiem jest tablica (nieposortowana) zawierająca poszukiwaną odpowiedź.


\end{document}
